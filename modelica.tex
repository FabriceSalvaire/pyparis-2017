
\begin{frame}
  \frametitle{Modelica: A language for modelling of complex physical systems}
  % What is Modelica ?
  Modelica in few words
  \begin{itemize}
  \item A \textbf{language} and a \textbf{Standard Library}
  \item Object-Oriented and Equation based
  \item Multi-domains
  \item Non-proprietary
  \item Developed by the non-profit \textbf{Modelica Association}
  \item initiated in September 1996 by Hilding Elmqvist
  \item First version on Sept. 1997  $\rightarrow$ 3.4 on April 2017
  \item Commercial front-ends: e.g.\@ Dymola
  \item Open Source front-ends: \textbf{OpenModelica} and \textbf{JModelica} \\[1em]
  \end{itemize}
  % \begin{columns}
  %   \begin{column}[t]{.65\textwidth}
  % \end{column}
  % \begin{column}[t]{.35\textwidth}
  %   Front-Ends
  %   \begin{itemize}
  %   \item Commercial : e.g.\@ Dymola
  %   \item Open Source : \\
  %     \begin{itemize}
  %     \item OpenModelica
  %     \item JModelica
  %     \end{itemize}
  %   \end{itemize}
  % \end{column}
  % \end{columns}
  {\tiny To go further \url{https://www.modelica.org}}
  \begin{textblock}{6}(13,7)
    \includegraphics[width=2cm]{images/modelica-logo.jpg}
  \end{textblock}
\end{frame}

\begin{frame}[fragile]
  \frametitle{Modelica: Bouncing-ball example}
  \fontsize{8pt}{8pt}\selectfont
\begin{Verbatim}[commandchars=\\\{\}]
\colorG{model} BouncingBall
  \colorG{parameter} Real e=0.7 \colorR{"coefficient of restitution"};
  \colorG{parameter} Real g=9.81 \colorR{"gravity acceleration"};
  Real h(start=1) \colorR{"height of ball"};
  Real v \colorR{"velocity of ball"};
  Boolean flying(start=true) \colorR{"true, if ball is flying"};
  Boolean impact;
  Real v_new;

\colorM{equation}
  impact = h <= 0;
  \colorB{der}(v) = \colorM{if} flying \colorM{then} -g \colorM{else} 0;
  \colorB{der}(h) = v;

  \colorO{// Triggered when one of theses conditions are true}
  \colorM{when} \{h <= 0 \colorM{and} v <= 0, impact\} \colorM{then}
    v_new = \colorM{if} \colorB{edge}(impact) \colorM{then} -e*\colorB{pre}(v) \colorM{else} 0;
    flying = v_new > 0;
    \colorB{reinit}(v, v_new);
  \colorM{end when};

\colorG{end} BouncingBall;
\end{Verbatim}
  \normalsize
\end{frame}

%%% Local Variables:
%%% mode: latex
%%% TeX-master: "master"
%%% End:


\begin{frame}
  \frametitle{Circuit Simulation: Nodal Analysis}
  \begin{columns}
    \begin{column}{.5\textwidth}
      \begin{center}
        \textbf{Circuit $\equiv$ Device's Graph} \\[1em]
        \includegraphics[width=1.\textwidth]{figures/network.pdf}
      \end{center}
    \end{column}
    \begin{column}{.5\textwidth}
      \textit{Roughly \ldots} \\[1em]
      \begin{enumerate}
      \item Apply \textbf{Kirchhoff's} circuit laws
        \begin{itemize}
        \item Kirchhoff's current law (KCL) \\
          $\sum {I}_k = 0$ for each node \\
          \textit{conservation of the electron flow} \\[.5em]
        \item Kirchhoff's voltage law (KVL) \\
          $\sum {V}_k = 0$ for each cycle \\
          \textit{think that is a closed path} \\[1em]
        \end{itemize}
      \item Apply \textbf{Device Equation} $f(V_k,I_k) = 0$ \\
        e.g.\@ Ohm Law for resistor $V = R I$ \\[.5em]
        Can be \textbf{Complex} e.g.\@ Capacitor, Inductor \\
        and \textbf{Non-Linear} e.g. Diode !
      \end{enumerate}
    \end{column}
  \end{columns}
\end{frame}

\begin{frame}
  \frametitle{Circuit Simulation: Nodal Analysis Example}
  \begin{columns}
    \begin{column}{.3\textwidth}
      \begin{center}
        \includegraphics[width=1.\textwidth]{figures/nodal-analysis.pdf}
      \end{center}
    \end{column}
    \begin{column}{.7\textwidth}
      Let apply the recipes: % Kirchhoff's circuit laws
      $$
      \begin{pmatrix}
        \frac{1}{R_1}+\frac{1}{R_2} & -\frac{1}{R_2} & 1 \\
        -\frac{1}{R_2} & \frac{1}{R_2}+\frac{1}{R_3} & -1 \\
        1 & -1 & 0
      \end{pmatrix}
      { \color{red!80}
      \begin{bmatrix}
        V_1 \\
        V_2 \\
        I_{V_{s1}}
      \end{bmatrix}
      }
      =
      \begin{bmatrix}
        I_{s1} \\
        0 \\
        V_{s1}
      \end{bmatrix}
      $$
      Then \textbf{solve} this \textbf{system of linear equations} \\[1em]
      There are \textbf{algorithms} to build theses \textbf{matrices} \\
      Usually matrices are \textbf{sparses} \\[1em]
      \textbf{Complex} therms for Capacitor, Inductor e.g.\@ $V^\star = \frac{1}{sC} I^\star$ % $V^\star = sL I^\star$
      \\[.5em]
      \textbf{Non linear} e.g.\@ Shockley Diode Model $I = I_s \left( e^{\frac{V}{n V_T}} - 1 \right)$
      \\[1em]
      {\tiny
        To go further \href{http://qucs.sourceforge.net/tech/technical.html}{QUCS Technical Papers}
        or \href{http://qucs.sourceforge.net/docs/technical/technical.pdf}{PDF}
      }
    \end{column}
  \end{columns}
\end{frame}

\begin{frame}
  \frametitle{Circuit Simulation: Several kinds of Analysis}
  \begin{columns}
    \begin{column}[t]{.6\textwidth}
      Principal Analyses \\[1em]
      \begin{itemize}
      \item DC Analysis \\
        \textit{Operating Point and DC Sweep} \\
        Compute node's voltages at $t=0$
        \textbf{cf. infra} \\[1em]
      \item AC Small-Signal Analysis \\
        \textbf{Transfer Function} \\
        Frequency Analysis $V(\omega)$, cf. Laplace Transform \\
        \textit{small means Taylor Series} \\[1em]
      \item Transient Analysis \\
        Simulate over time $V(t)$ \\
        \textbf{Integrator}
      \end{itemize}
    \end{column}
    \begin{column}[t]{.4\textwidth}
      More specific ones \\[1em]
      \begin{itemize}
      \item Pole-Zero Analysis
      \item Small-Signal Distortion Analysis
      \item Sensitivity Analysis
      \item Noise Analysis
      \item etc.\@
      \end{itemize}
    \end{column}
  \end{columns}
  \begin{textblock}{6}(11,8)
    {\tiny
      To go further \href{http://ngspice.sourceforge.net/docs.html}{Ngspice Documentation}
      % \href{http://ngspice.sourceforge.net/docs/ngspice-manual.pdf}{Ngspice user's manual}
    }
  \end{textblock}
\end{frame}

%%% Local Variables:
%%% mode: latex
%%% TeX-master: "master"
%%% End:

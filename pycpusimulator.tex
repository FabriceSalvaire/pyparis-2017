
\begin{frame}
  \frametitle{PyCpuSimulator: Simulate CPU using micro-code}
  \begin{center}
    \url{https://github.com/FabriceSalvaire/PyCpuSimulator} \\[2em]
    \begin{tikzpicture}%
      [node distance=10mm,
      start chain=going right,
      >=stealth',
      punktchain/.style={
        rectangle,
        rounded corners,
        % fill=black!10,
        draw=black, very thick,
        text width=7em,
        minimum height=3em,
        text centered,
        on chain},
      line/.style={draw, thick, <-},
      every join/.style={->, thick,shorten >=1pt},
      ]
      \node[punktchain, join] {CPU Datasheet Extractor};
      \node[punktchain, join] {YAML Instruction Set};
      \node[punktchain, join] {Micro-Code Interpreter};
    \end{tikzpicture}
    \vspace{1em}
  \end{center}
  \begin{columns}
    \begin{column}{.6\textwidth}
      Features
      \begin{itemize}
      \item Micro-Code Language to describe instruction
      \item Opcode Decoder using Decision Tree
      \item Read HEX firmware format
      \item AVR Core CPU simulation is ongoing
      \end{itemize}
    \end{column}
    \begin{column}{.4\textwidth}
      {\small \alert{If you want to play with Core \ldots}}
    \end{column}
  \end{columns}
\end{frame}

%%% Local Variables:
%%% mode: latex
%%% TeX-master: "master"
%%% End:
